% Colores locales
\usepackage{xcolor}
\definecolor{ctitbox}{HTML}{E6E6FF}
\definecolor{capnum}{HTML}{0606FF}

% Imagenes
\usepackage{graphicx}
\newcommand{\imagen}[2]{
	\begin{figure}[!h]
		\centering
		\includegraphics[width=0.9\textwidth]{#1}
		\caption{#2}\label{fig:#1}
	\end{figure}
	\FloatBarrier
}

\newcommand{\imagenflotante}[2]{
	\begin{figure}%[!h]
		\centering
		\includegraphics[width=0.9\textwidth]{#1}
		\caption{#2}\label{fig:#1}
	\end{figure}
}


% El comando \figura nos permite insertar figuras comodamente, y utilizando
% siempre el mismo formato. Los parametros son:
% 1 -> Porcentaje del ancho de página que ocupará la figura (de 0 a 1)
% 2 --> Fichero de la imagen
% 3 --> Texto a pie de imagen
% 4 --> Etiqueta (label) para referencias
% 5 --> Opciones que queramos pasarle al \includegraphics
% 6 --> Opciones de posicionamiento a pasarle a \begin{figure}
\newcommand{\figuraConPosicion}[6]{%
  \setlength{\anchoFloat}{#1\textwidth}%
  \addtolength{\anchoFloat}{-4\fboxsep}%
  \setlength{\anchoFigura}{\anchoFloat}%
  \begin{figure}[#6]
    \begin{center}%
      \Ovalbox{%
        \begin{minipage}{\anchoFloat}%
          \begin{center}%
            \includegraphics[width=\anchoFigura,#5]{#2}%
            \caption{#3}%
            \label{#4}%
          \end{center}%
        \end{minipage}
      }%
    \end{center}%
  \end{figure}%
}

%
% Comando para incluir imágenes en formato apaisado (sin marco).
\newcommand{\figuraApaisadaSinMarco}[5]{%
  \begin{figure}%
    \begin{center}%
    \includegraphics[angle=90,height=#1\textheight,#5]{#2}%
    \caption{#3}%
    \label{#4}%
    \end{center}%
  \end{figure}%
}
% Para las tablas
\newcommand{\otoprule}{\midrule [\heavyrulewidth]}
%
% Nuevo comando para tablas pequeñas (menos de una página).
\newcommand{\tablaSmall}[5]{%
 \begin{table}
  \begin{center}
   \rowcolors {2}{gray!35}{}
   \begin{tabular}{#2}
    \toprule
    #4
    \otoprule
    #5
    \bottomrule
   \end{tabular}
   \caption{#1}
   \label{tabla:#3}
  \end{center}
 \end{table}
}

%
% Nuevo comando para tablas pequeñas (menos de una página).
\newcommand{\tablaSmallSinColores}[5]{%
 \begin{table}[H]
  \begin{center}
   \begin{tabular}{#2}
    \toprule
    #4
    \otoprule
    #5
    \bottomrule
   \end{tabular}
   \caption{#1}
   \label{tabla:#3}
  \end{center}
 \end{table}
}

\newcommand{\tablaApaisadaSmall}[5]{%
\begin{landscape}
  \begin{table}
   \begin{center}
    \rowcolors {2}{gray!35}{}
    \begin{tabular}{#2}
     \toprule
     #4
     \otoprule
     #5
     \bottomrule
    \end{tabular}
    \caption{#1}
    \label{tabla:#3}
   \end{center}
  \end{table}
\end{landscape}
}

%
% Nuevo comando para tablas grandes con cabecera y filas alternas coloreadas en gris.
\newcommand{\tabla}[6]{%
  \begin{center}
    \tablefirsthead{
      \toprule
      #5
      \otoprule
    }
    \tablehead{
      \multicolumn{#3}{l}{\small\sl continúa desde la página anterior}\\
      \toprule
      #5
      \otoprule
    }
    \tabletail{
      \hline
      \multicolumn{#3}{r}{\small\sl continúa en la página siguiente}\\
    }
    \tablelasttail{
      \hline
    }
    \bottomcaption{#1}
    \rowcolors {2}{gray!35}{}
    \begin{xtabular}{#2}
      #6
      \bottomrule
    \end{xtabular}
    \label{tabla:#4}
  \end{center}
}

%
% Nuevo comando para tablas grandes con cabecera.
\newcommand{\tablaSinColores}[6]{%
  \begin{center}
    \tablefirsthead{
      \toprule
      #5
      \otoprule
    }
    \tablehead{
      \multicolumn{#3}{l}{\small\sl continúa desde la página anterior}\\
      \toprule
      #5
      \otoprule
    }
    \tabletail{
      \hline
      \multicolumn{#3}{r}{\small\sl continúa en la página siguiente}\\
    }
    \tablelasttail{
      \hline
    }
    \bottomcaption{#1}
    \begin{xtabular}{#2}
      #6
      \bottomrule
    \end{xtabular}
    \label{tabla:#4}
  \end{center}
}

%
% Nuevo comando para tablas grandes sin cabecera.
\newcommand{\tablaSinCabecera}[5]{%
  \begin{center}
    \tablefirsthead{
      \toprule
    }
    \tablehead{
      \multicolumn{#3}{l}{\small\sl continúa desde la página anterior}\\
      \hline
    }
    \tabletail{
      \hline
      \multicolumn{#3}{r}{\small\sl continúa en la página siguiente}\\
    }
    \tablelasttail{
      \hline
    }
    \bottomcaption{#1}
  \begin{xtabular}{#2}
    #5
   \bottomrule
  \end{xtabular}
  \label{tabla:#4}
  \end{center}
}



\definecolor{cgoLight}{HTML}{EEEEEE}
\definecolor{cgoExtralight}{HTML}{FFFFFF}

%
% Nuevo comando para tablas grandes sin cabecera.
\newcommand{\tablaSinCabeceraConBandas}[5]{%
  \begin{center}
    \tablefirsthead{
      \toprule
    }
    \tablehead{
      \multicolumn{#3}{l}{\small\sl continúa desde la página anterior}\\
      \hline
    }
    \tabletail{
      \hline
      \multicolumn{#3}{r}{\small\sl continúa en la página siguiente}\\
    }
    \tablelasttail{
      \hline
    }
    \bottomcaption{#1}
    \rowcolors[]{1}{cgoExtralight}{cgoLight}

  \begin{xtabular}{#2}
    #5
   \bottomrule
  \end{xtabular}
  \label{tabla:#4}
  \end{center}
}


\graphicspath{ {./img/} }

% Capítulos
\chapterstyle{bianchi}
\newcommand{\capitulo}[2]{
	\setcounter{chapter}{#1}
	\setcounter{section}{0}
	\cleardoublepage\chapter*{{\color{blue} #1:} #2}
	\addcontentsline{toc}{chapter}{#1. #2}
	\markboth{#2}{#2}
}

% Apéndices
\renewcommand{\appendixname}{Apéndice}
\renewcommand*\cftappendixname{\appendixname}

\newcommand{\apendice}[1]{
	%\renewcommand{\thechapter}{A}
	\chapter{#1}
}

\renewcommand*\cftappendixname{\appendixname\ }

% Formato de portada
\makeatletter
\newcommand{\tutor}[1]{\def\@tutor{#1}}
\newcommand{\course}[1]{\def\@course{#1}}

%\usepackage{setspace}
\def\maketitle{
  \null
  \thispagestyle{empty}
  % Cabecera ----------------
    \begin{center}%
    	{\noindent\Huge \textsc{Universidad de Valladolid}}\\[.3cm]%
    	{\noindent\Large \textsc{Máster universitario}}\\[.3cm]%
    	{\noindent\Huge \textbf{Ingeniería Informática}}\\[1cm]%
    
    	\includegraphics[height=3cm]{img/escudoUVA} \hfill
    	\includegraphics[height=3cm]{img/escudoInfor.pdf} \hfill
    	\includegraphics[height=3cm]{img/escudoEIIVA.png} \hfill
    \end{center}%
  \vfill
    \begin{center}%
      % Título proyecto y escudo informática ----------------
      {\noindent\large\textsc{Trabajo Fin de Máster}}\\[.3cm]
      \colorbox{ctitbox}{%
        \begin{minipage}{.95\textwidth}
          \vspace{.2cm}\large
        %   \begin{center}
        %   \textsc{Máster en Ingeniería Informática}
        %   \end{center}
          \begin{center}
%          \renewcommand{\baselinestretch}{2.0}
          \textbf{\LARGE\@title{}}
%          \renewcommand{\baselinestretch}{1.0}
          \end{center}
          \vspace{.2cm}
        \end{minipage}
      }%
    \end{center}%
  \hfill
  \vfill
  % Datos de alumno, curso y tutores ------------------
  \begin{center}%
  {%
    \noindent\large
    Realizado por \Large\textbf{\@author{}}\\[0.2cm]
    \maltese\hspace{.1cm}\maltese\hspace{.1cm}\maltese\\[0.2cm]
    {\Large\sffamily\bfseries Universidad de Valladolid}\\
    \textbf{\@date{}}\\
    Tutor: \@tutor{}\\
  }%
  \end{center}%
  \null
  \cleardoublepage
}
\makeatother
