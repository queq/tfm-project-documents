\capitulo{1}{Introducción}

En un mundo cada vez más dependiente de las tecnologías de la información y comunicaciones, el acontecimiento de ciberataques trae mayores riesgos y posible impacto negativo en las sociedades. Como evidencia, 2023 fue un año en que ocurrieron importantes filtraciones de datos masivas, e incidentes que comprometieron el funcionamiento de entidades de ámbito nacional \cite{moore2023}.

Entre las amenazas principales, el \textit{phishing} se ha mantenido como el vector de acceso inicial más común por su inherente relación con la comunicación humana, y por la diversidad de métodos de ataque, los cuales se han vuelto más sofisticados ante la aparición de nuevos avances, como la IA generativa \cite{enisa2023}. El medio más usual para transmitir ataques de \textit{phishing} es a través de páginas web, ya que sus URL pueden ser incrustadas en cualquier tipo de mensajería virtual. Por lo anterior, se evidencia la relevancia en contribuir con la investigación de herramientas para la detección de \textit{phishing} en URLs.

\section{Motivación}

Citar surveys o compilados de fuentes de datos de phishing para apoyar la motivación en crear un corpus con mayor volumen.

\section{Objetivos}

Desarrollar un corpus y un marco de trabajo para selección y aplicación de diferentes técnicas de \textit{Machine Learning} en el contexto de detección y prevención de \textit{phishing} en páginas web.

Objetivos específicos

\begin{itemize}
    \item Definir los conjuntos de datos más relevantes en el ámbito académico de la investigación en detección de páginas de \textit{phishing} en funcionamiento actualmente.
    \item Construir un marco de trabajo (\textit{framework}) para facilitar la recopilación de datos de páginas de \textit{phishing} para uso en investigación.
    \item Exponer una selección de atributos para aplicar en detección basada en \textit{Machine Learning}, basados en el estado del arte.
    \item Componer un conjunto de datos curado (\textit{corpus}) mediante uso del \textit{framework} implementado, que incluya las fuentes de datos primarias definidas y la selección de atributos planteada.
    \item Determinar y caracterizar las principales técnicas de detección de páginas de \textit{phishing} basadas en \textit{Machine Learning}, investigadas en los últimos 5 años.
\end{itemize}

\section{Estructura del documento}